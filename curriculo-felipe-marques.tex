%-----------------------------------------------------------------------------------------------------------------------------------------------%
%	The MIT License (MIT)
%
%	Copyright (c) 2021 Jitin Nair
%
%	Permission is hereby granted, free of charge, to any person obtaining a copy
%	of this software and associated documentation files (the "Software"), to deal
%	in the Software without restriction, including without limitation the rights
%	to use, copy, modify, merge, publish, distribute, sublicense, and/or sell
%	copies of the Software, and to permit persons to whom the Software is
%	furnished to do so, subject to the following conditions:
%	
%	THE SOFTWARE IS PROVIDED "AS IS", WITHOUT WARRANTY OF ANY KIND, EXPRESS OR
%	IMPLIED, INCLUDING BUT NOT LIMITED TO THE WARRANTIES OF MERCHANTABILITY,
%	FITNESS FOR A PARTICULAR PURPOSE AND NONINFRINGEMENT. IN NO EVENT SHALL THE
%	AUTHORS OR COPYRIGHT HOLDERS BE LIABLE FOR ANY CLAIM, DAMAGES OR OTHER
%	LIABILITY, WHETHER IN AN ACTION OF CONTRACT, TORT OR OTHERWISE, ARISING FROM,
%	OUT OF OR IN CONNECTION WITH THE SOFTWARE OR THE USE OR OTHER DEALINGS IN
%	THE SOFTWARE.
%	
%
%-----------------------------------------------------------------------------------------------------------------------------------------------%

%----------------------------------------------------------------------------------------
%	DOCUMENT DEFINITION
%----------------------------------------------------------------------------------------

% article class because we want to fully customize the page and not use a cv template
\documentclass[a4paper,12pt]{article}

%----------------------------------------------------------------------------------------
%	FONT
%----------------------------------------------------------------------------------------

% % fontspec allows you to use TTF/OTF fonts directly
% \usepackage{fontspec}
% \defaultfontfeatures{Ligatures=TeX}

% % modified for ShareLaTeX use
% \setmainfont[
% SmallCapsFont = Fontin-SmallCaps.otf,
% BoldFont = Fontin-Bold.otf,
% ItalicFont = Fontin-Italic.otf
% ]
% {Fontin.otf}

\usepackage[T1]{fontenc}
\usepackage{lmodern}

%----------------------------------------------------------------------------------------
%	PACKAGES
%----------------------------------------------------------------------------------------
\usepackage{url}
\usepackage{parskip} 	

%other packages for formatting
\RequirePackage{color}
\RequirePackage{graphicx}
\usepackage[usenames,dvipsnames]{xcolor}
\usepackage[scale=0.9]{geometry}

%tabularx environment
\usepackage{tabularx}

%for lists within experience section
\usepackage{enumitem}

% centered version of 'X' col. type
\newcolumntype{C}{>{\centering\arraybackslash}X} 

%to prevent spillover of tabular into next pages
\usepackage{supertabular}
\usepackage{tabularx}
\newlength{\fullcollw}
\setlength{\fullcollw}{0.47\textwidth}

%custom \section
\usepackage{titlesec}				
\usepackage{multicol}
\usepackage{multirow}

%CV Sections inspired by: 
%http://stefano.italians.nl/archives/26
\titleformat{\section}{\Large\scshape\raggedright}{}{0em}{}[\titlerule]
\titlespacing{\section}{0pt}{10pt}{10pt}

%for publications
\usepackage[style=authoryear,sorting=ynt, maxbibnames=2]{biblatex}

%Setup hyperref package, and colours for links
\usepackage[unicode, draft=false]{hyperref}
\definecolor{linkcolour}{rgb}{0,0.2,0.6}
\hypersetup{colorlinks,breaklinks,urlcolor=linkcolour,linkcolor=linkcolour}
\addbibresource{citations.bib}
\setlength\bibitemsep{1em}

%for social icons
\usepackage{fontawesome5}

% Para tradução do /today
\usepackage[portuguese]{babel}

% #1 Título | #2 Período | #3 Meta/observação (ex.: Tempo integral; Fullstack)
\newenvironment{jobshort}[3]
    {
    \begin{tabularx}{\linewidth}{@{}l X r@{}}
    \textbf{#1}\ \ {\footnotesize\if\relax\detokenize{#3}\relax\else #3\fi} & \hfill & #2 \\
    \end{tabularx}
    }
    {
    }

% #1 Título | 2# Meta/observação (ex.: Remoto)
\newenvironment{joblong}[2]{
  \begin{tabularx}{\linewidth}{@{}l X@{}}
    \textbf{#1} & \if\relax\detokenize{#2}\relax\else #2\fi \\
  \end{tabularx}
  \begin{itemize}[leftmargin=1.5em,label={},itemsep=5pt,topsep=0pt]
}{
  \end{itemize}
}

% #1 Título | #2 Período | #3 Meta/observação (ex.: Tempo integral; Fullstack)
\newenvironment{role}[3]{
  \item
  \begin{tabularx}{\linewidth}{@{}l X r@{}}
    {\small\textbf{#1}}\ \ {\footnotesize\if\relax\detokenize{#3}\relax\else #3\fi} & \hfill & #2 \\
  \end{tabularx}
    }
    {
    }

\newenvironment{education}[3]
    {
    \begin{tabularx}{\linewidth}{@{}l X r@{}}
    \textbf{#1} - \textbf{#2} & \hfill &  #3 \\[3.75pt]
    \end{tabularx}
    }
    {
    }

%----------------------------------------------------------------------------------------
%	BEGIN DOCUMENT
%----------------------------------------------------------------------------------------
\begin{document}

% non-numbered pages
\pagestyle{empty} 

%----------------------------------------------------------------------------------------
%	TITLE
%----------------------------------------------------------------------------------------

\begin{tabularx}{\linewidth}{@{} C @{}}
\Huge{Felipe da Costa Marques} \\[7.5pt]

\href{https://felipemarques.dev.br/}{\raisebox{-0.05\height}\faGlobe\ felipemarques.dev.br}
\ $|$ \
\href{mailto:felipe.marques.desenvolvedor@gmail.com}{\raisebox{-0.05\height}\faEnvelope\ felipe.marques.desenvolvedor@gmail.com} \\

\href{https://github.com/helpmeagain}{\raisebox{-0.05\height}\faGithub\ helpmeagain}
\ $|$ \
\href{https://www.linkedin.com/in/felipecomarques/}{\raisebox{-0.05\height}\faLinkedin\ felipecomarques} 

\end{tabularx}

%----------------------------------------------------------------------------------------
% EXPERIENCE SECTIONS
%----------------------------------------------------------------------------------------

\section{Objetivos}
Engenheiro de software com experiência em desenvolvimento de aplicações web full stack e mobile, com foco em arquitetura de sistemas distribuídos. Busco oportunidades para atuar no desenvolvimento de soluções escaláveis e bem estruturadas, utilizando boas práticas de engenharia de software e tecnologias modernas.

\section{Experiência profissional}

\begin{jobshort}{Compass UOL}{Maio 2024 — Jan 2025}{Trainee, Fullstack}
Atuei no desenvolvimento e manutenção de sistemas voltados à gestão de aplicativos de telecomunicações, contribuindo para a implementação de novas funcionalidades em microserviços, criação de relatórios de usuários, adaptação de interfaces e realização de releases em ambiente de produção. Essa atuação resultou no aumento da estabilidade das aplicações e na otimização do fluxo de entrega de novas funcionalidades.
\begin{itemize}[nosep,after=\strut, leftmargin=0em, itemsep=3pt,label={}]
    \item \textbf{Stack}: JavaScript, TypeScript, NestJS, Express, Angular, Ionic, MongoDB, Redis, Azure DevOps.
\end{itemize}
\end{jobshort}

\begin{jobshort}{Compass UOL}{Maio 2023 — Out 2023}{Estagiário, Backend}
Atuei em um ambiente ágil utilizando Scrum, participando do desenvolvimento e manutenção de APIs REST com Node.js, TypeScript, Express e MongoDB. Colaborei na implementação de autenticação, segurança, documentação e endpoints, além da criação de testes unitários e de integração. Também utilizei serviços AWS para o deploy e a manutenção da aplicação em ambiente de produção.
\begin{itemize}[nosep,after=\strut, leftmargin=0em, itemsep=3pt,label={}]
    \item \textbf{Stack}: TypeScript, Express, MongoDB, Docker, AWS.
\end{itemize}
\end{jobshort}

% \begin{joblong}{Compass UOL}{}
%   \begin{role}{Desenvolvedor Trainee}{Maio 2024 — Jan 2025}{Backend}
%     Atuei contribuindo para o desenvolvimento e manutenção de sistemas voltados à gestão de conectividade digital. Fui responsável pela implementação de novas funcionalidades em microserviços, criação de relatórios de usuários, adaptação de interfaces e realização de releases em ambiente produtivo. Minha atuação contribuiu para melhorar a estabilidade das aplicações e otimizar o fluxo de entrega de novas funcionalidades.
%     \begin{itemize}[nosep,after=\strut, leftmargin=0em, itemsep=3pt,label={}]
%     % \begin{itemize}[nosep,after=\strut, leftmargin=1em, itemsep=3pt,label=--]
%         \item Stack: JavaScript, TypeScript, NestJS, Express, Angular, Ionic, MongoDB, Redis, Azure DevOps.
%     \end{itemize}
%   \end{role}

%   \begin{role}{Estagiário}{Maio 2023 — Out 2023}{Backend}
%     Atuei em um ambiente ágil com Scrum, participando do desenvolvimento e manutenção de APIs REST em Node.js, TypeScript, Express e MongoDB. Colaborei na implementação de autenticação, segurança e na documentação com Swagger, além de realizar testes unitários e de integração. Explorei serviços AWS, fortalecendo conhecimentos em cloud e arquitetura de software.
%     \begin{itemize}[nosep,after=\strut, leftmargin=0em, itemsep=3pt,label={}]
%         \item Stack: TypeScript, Express, MongoDB, Docker, AWS.
%     \end{itemize}
%   \end{role}
% \end{joblong}
  

\section{Formação}
    
\begin{education}{Bacharelado em Sistemas de Informação}{IFCE Campus Cedro}{2021 — 2026}
Estrutura de Dados, Engenharia de Software, Programação Web, Gerência de Projetos.
\end{education}

\begin{education}{Ensino Médio Integrado ao Técnico em Informática}{IFCE Campus Cedro}{2017 — 2020}
Lógica de Programação, Banco de Dados, Programação Orientada a Objeto.
\end{education}


\section{Projetos}

\begin{tabularx}{\linewidth}{ @{}l r@{} }
\textbf{Weather Dashboard} & \hfill \href{https://youtu.be/OMvqebhPMrA}{Demonstração} — \href{https://github.com/helpmeagain/weather-dashboard}{Repositório} \\[3.75pt]
\multicolumn{2}{@{}X@{}}{Dashboard desenvolvido para visualização e análise de dados climáticos, estruturado em uma arquitetura baseada em microsserviços. O sistema realiza a coleta periódica de dados meteorológicos, processamento intermediário, persistência e geração de insights por IA, oferecendo uma visão interativa ao usuário final.}  \\[-2pt]
\multicolumn{2}{@{}X@{}}{\vspace{1pt}\textbf{Stack}: Python, Go, TypeScript, NestJS, React, Vite, MongoDB, RabbitMQ, Ollama, Docker.}\\[1pt]
\end{tabularx}

\begin{tabularx}{\linewidth}{ @{}l r@{} }
\textbf{Project Kiwi} & \hfill \href{https://helpmeagain.itch.io/project-kiwi}{Demonstração} — \href{https://github.com/helpmeagain/project-kiwi}{Repositório} \\[3.75pt]
\multicolumn{2}{@{}X@{}}{Jogo educacional para ensino de inglês com o objetivo de tornar o aprendizado gamificado mais colaborativo e interativo. O projeto foi implementado com foco em experiência do usuário, jogabilidade em duplas e usabilidade. Os testes com estudantes mostraram que a adição do modo multijogador aumentou o engajamento, a aceitação e o interesse pelo conteúdo.}  \\[-2pt]
\multicolumn{2}{@{}X@{}}{\vspace{1pt}\textbf{Stack}: Godot, GDScript, Nakama, Docker.}\\[1pt]
\end{tabularx}

\begin{tabularx}{\linewidth}{ @{}l r@{} }
\textbf{Blackwell} & \hfill \href{https://github.com/helpmeagain/blackwell}{Repositório} \\[3.75pt]
\multicolumn{2}{@{}X@{}}{API e aplicação mobile desenvolvida para geração de prontuários eletrônicos de fisioterapia pra uma clínica-escola. A solução foi estruturada com Domain-Driven Design e Clean Architecture, garantindo escalabilidade e facilidade de manutenção. O projeto otimizou o fluxo de atendimento, reduziu erros no registro de dados e agilizou o acesso às informações dos pacientes.}  \\[-2pt]
\multicolumn{2}{@{}X@{}}{\vspace{1pt}\textbf{Stack}: TypeScript, NestJS, Prisma, Vitest, PostgreSQL, Redis, Docker, React Native, Expo.}\\[1pt]
\end{tabularx}

%----------------------------------------------------------------------------------------
%	PUBLICATIONS
%----------------------------------------------------------------------------------------
% \section{Publicações}
% \begin{refsection}[citations.bib]
% \nocite{*}
% \printbibliography[heading=none]
% \end{refsection}

%----------------------------------------------------------------------------------------
%	SKILLS
%----------------------------------------------------------------------------------------
\section{Habilidades}
\begin{tabularx}{\linewidth}{@{}l X@{}}
\textbf{Linguagens de programação} & \normalsize{JavaScript, TypeScript, C{\texttt{\#}}, Go, Java, Python.}\\
\textbf{Desenvolvimento Web} & \normalsize{NodeJS, Express, NestJS, React, NextJS, Angular, VueJS, NuxtJS, HTML, CSS, Tailwind, .NET, Entity Framework, Django, Flask, Springboot.}\\
\textbf{Desenvolvimento Mobile} & \normalsize{React Native, Expo, Ionic.}\\
\textbf{Banco de dados}  &  \normalsize{MySQL, SQL Server, PostgreSQL, MongoDB, Redis.}\\
\textbf{Miscelânea}  &  \normalsize{Docker, AWS, Google Cloud, Azure DevOps, Ollama, Godot.}\\  
\end{tabularx}

\vfill
\center{\footnotesize Última atualização: \today}

\end{document}

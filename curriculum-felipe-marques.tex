%-----------------------------------------------------------------------------------------------------------------------------------------------%
%	The MIT License (MIT)
%
%	Copyright (c) 2021 Jitin Nair
%
%	Permission is hereby granted, free of charge, to any person obtaining a copy
%	of this software and associated documentation files (the "Software"), to deal
%	in the Software without restriction, including without limitation the rights
%	to use, copy, modify, merge, publish, distribute, sublicense, and/or sell
%	copies of the Software, and to permit persons to whom the Software is
%	furnished to do so, subject to the following conditions:
%	
%	THE SOFTWARE IS PROVIDED "AS IS", WITHOUT WARRANTY OF ANY KIND, EXPRESS OR
%	IMPLIED, INCLUDING BUT NOT LIMITED TO THE WARRANTIES OF MERCHANTABILITY,
%	FITNESS FOR A PARTICULAR PURPOSE AND NONINFRINGEMENT. IN NO EVENT SHALL THE
%	AUTHORS OR COPYRIGHT HOLDERS BE LIABLE FOR ANY CLAIM, DAMAGES OR OTHER
%	LIABILITY, WHETHER IN AN ACTION OF CONTRACT, TORT OR OTHERWISE, ARISING FROM,
%	OUT OF OR IN CONNECTION WITH THE SOFTWARE OR THE USE OR OTHER DEALINGS IN
%	THE SOFTWARE.
%	
%
%-----------------------------------------------------------------------------------------------------------------------------------------------%

%----------------------------------------------------------------------------------------
%	DOCUMENT DEFINITION
%----------------------------------------------------------------------------------------

% article class because we want to fully customize the page and not use a cv template
\documentclass[a4paper,12pt]{article}

%----------------------------------------------------------------------------------------
%	FONT
%----------------------------------------------------------------------------------------

% % fontspec allows you to use TTF/OTF fonts directly
% \usepackage{fontspec}
% \defaultfontfeatures{Ligatures=TeX}

% % modified for ShareLaTeX use
% \setmainfont[
% SmallCapsFont = Fontin-SmallCaps.otf,
% BoldFont = Fontin-Bold.otf,
% ItalicFont = Fontin-Italic.otf
% ]
% {Fontin.otf}

%----------------------------------------------------------------------------------------
%	PACKAGES
%----------------------------------------------------------------------------------------
\usepackage{url}
\usepackage{parskip} 	

%other packages for formatting
\RequirePackage{color}
\RequirePackage{graphicx}
\usepackage[usenames,dvipsnames]{xcolor}
\usepackage[scale=0.9]{geometry}

%tabularx environment
\usepackage{tabularx}

%for lists within experience section
\usepackage{enumitem}

% centered version of 'X' col. type
\newcolumntype{C}{>{\centering\arraybackslash}X} 

%to prevent spillover of tabular into next pages
\usepackage{supertabular}
\usepackage{tabularx}
\newlength{\fullcollw}
\setlength{\fullcollw}{0.47\textwidth}

%custom \section
\usepackage{titlesec}				
\usepackage{multicol}
\usepackage{multirow}

%CV Sections inspired by: 
%http://stefano.italians.nl/archives/26
\titleformat{\section}{\Large\scshape\raggedright}{}{0em}{}[\titlerule]
\titlespacing{\section}{0pt}{10pt}{10pt}

%for publications
\usepackage[style=authoryear,sorting=ynt, maxbibnames=2]{biblatex}

%Setup hyperref package, and colours for links
\usepackage[unicode, draft=false]{hyperref}
\definecolor{linkcolour}{rgb}{0,0.2,0.6}
\hypersetup{colorlinks,breaklinks,urlcolor=linkcolour,linkcolor=linkcolour}
\addbibresource{citations.bib}
\setlength\bibitemsep{1em}

%for social icons
\usepackage{fontawesome5}

%debug page outer frames
%\usepackage{showframe}

% #1 Título | #2 Período | #3 Meta/observação (ex.: Tempo integral; Fullstack)
\newenvironment{jobshort}[3]
    {
    \begin{tabularx}{\linewidth}{@{}l X r@{}}
    \textbf{#1}\ \ {\footnotesize\if\relax\detokenize{#3}\relax\else #3\fi} & \hfill & #2 \\
    \end{tabularx}
    }
    {
    }

% #1 Título | 2# Meta/observação (ex.: Remoto)
\newenvironment{joblong}[2]{
  \begin{tabularx}{\linewidth}{@{}l X@{}}
    \textbf{#1} & \if\relax\detokenize{#2}\relax\else #2\fi \\
  \end{tabularx}
  \begin{itemize}[leftmargin=1.5em,label={},itemsep=5pt,topsep=0pt]
}{
  \end{itemize}
}

% #1 Título | #2 Período | #3 Meta/observação (ex.: Tempo integral; Fullstack)
\newenvironment{role}[3]{
  \item
  \begin{tabularx}{\linewidth}{@{}l X r@{}}
    {\small\textbf{#1}}\ \ {\footnotesize\if\relax\detokenize{#3}\relax\else #3\fi} & \hfill & #2 \\
  \end{tabularx}
    }
    {
    }

\newenvironment{education}[3]
    {
    \begin{tabularx}{\linewidth}{@{}l X r@{}}
    \textbf{#1} - \textbf{#2} & \hfill &  #3 \\[3.75pt]
    \end{tabularx}
    }
    {
    }

%----------------------------------------------------------------------------------------
%	BEGIN DOCUMENT
%----------------------------------------------------------------------------------------
\begin{document}

% non-numbered pages
\pagestyle{empty} 

%----------------------------------------------------------------------------------------
%	TITLE
%----------------------------------------------------------------------------------------

\begin{tabularx}{\linewidth}{@{} C @{}}
\Huge{Felipe da Costa Marques} \\[7.5pt]

\href{https://felipemarques.dev.br/}{\raisebox{-0.05\height}\faGlobe\ felipemarques.dev.br}
\ $|$ \
\href{mailto:felipe.marques.desenvolvedor@gmail.com}{\raisebox{-0.05\height}\faEnvelope\ felipe.marques.desenvolvedor@gmail.com} \\

\href{https://github.com/helpmeagain}{\raisebox{-0.05\height}\faGithub\ helpmeagain}
\ $|$ \
\href{https://www.linkedin.com/in/felipecomarques/}{\raisebox{-0.05\height}\faLinkedin\ felipecomarques} 

\end{tabularx}

%----------------------------------------------------------------------------------------
% EXPERIENCE SECTIONS
%----------------------------------------------------------------------------------------

\section{Objective}
Software engineer with experience in full stack web and mobile application development, with a focus on distributed systems architecture. Seeking opportunities to work on the development of scalable and well-structured solutions, applying software engineering best practices and modern technologies.

\section{Professional Experience}

\begin{jobshort}{Compass UOL}{May 2024 — Jan 2025}{Trainee, Full Stack}
Worked on the development and maintenance of systems focused on the management of telecommunications applications, contributing to the implementation of new features in microservices, creation of user reports, interface adaptations, and execution of production releases. This work resulted in increased application stability and optimization of the feature delivery workflow.
\begin{itemize}[nosep,after=\strut, leftmargin=0em, itemsep=3pt,label={}]
    \item \textbf{Stack}: JavaScript, TypeScript, NestJS, Express, Angular, Ionic, MongoDB, Redis, Azure DevOps.
\end{itemize}
\end{jobshort}

\begin{jobshort}{Compass UOL}{May 2023 — Oct 2023}{Backend Intern}
Worked in an agile environment using Scrum, contributing to the development and maintenance of REST APIs with Node.js, TypeScript, Express, and MongoDB. Collaborated on the implementation of authentication, security, documentation, and endpoints, as well as the creation of unit and integration tests. Also utilized AWS services for application deployment and maintenance in a production environment.
\begin{itemize}[nosep,after=\strut, leftmargin=0em, itemsep=3pt,label={}]
    \item \textbf{Stack}: TypeScript, Express, MongoDB, Docker, AWS.
\end{itemize}
\end{jobshort}

% \begin{joblong}{Compass UOL}{}
%   \begin{role}{Desenvolvedor Trainee}{Maio 2024 — Jan 2025}{Backend}
%     Atuei contribuindo para o desenvolvimento e manutenção de sistemas voltados à gestão de conectividade digital. Fui responsável pela implementação de novas funcionalidades em microserviços, criação de relatórios de usuários, adaptação de interfaces e realização de releases em ambiente produtivo. Minha atuação contribuiu para melhorar a estabilidade das aplicações e otimizar o fluxo de entrega de novas funcionalidades.
%     \begin{itemize}[nosep,after=\strut, leftmargin=0em, itemsep=3pt,label={}]
%     % \begin{itemize}[nosep,after=\strut, leftmargin=1em, itemsep=3pt,label=--]
%         \item Stack: JavaScript, TypeScript, NestJS, Express, Angular, Ionic, MongoDB, Redis, Azure DevOps.
%     \end{itemize}
%   \end{role}

%   \begin{role}{Estagiário}{Maio 2023 — Out 2023}{Backend}
%     Atuei em um ambiente ágil com Scrum, participando do desenvolvimento e manutenção de APIs REST em Node.js, TypeScript, Express e MongoDB. Colaborei na implementação de autenticação, segurança e na documentação com Swagger, além de realizar testes unitários e de integração. Explorei serviços AWS, fortalecendo conhecimentos em cloud e arquitetura de software.
%     \begin{itemize}[nosep,after=\strut, leftmargin=0em, itemsep=3pt,label={}]
%         \item Stack: TypeScript, Express, MongoDB, Docker, AWS.
%     \end{itemize}
%   \end{role}
% \end{joblong}
  

\section{Education}
    
\begin{education}{Bachelor’s Degree in Information Systems}{IFCE Campus Cedro}{2021 — 2026}
Data Structures, Software Engineering, Web Programming, Project Management.
\end{education}

\begin{education}{Technical High School in Information Technology}{IFCE Campus Cedro}{2017 — 2020}
Programming Logic, Databases, Object-Oriented Programming.
\end{education}


\section{Projects}

\begin{tabularx}{\linewidth}{ @{}l r@{} }
\textbf{Weather Dashboard} & \hfill \href{https://youtu.be/OMvqebhPMrA}{Demo} — \href{https://github.com/helpmeagain/weather-dashboard}{Repository} \\[3.75pt]
\multicolumn{2}{@{}X@{}}{Dashboard developed for visualization and analysis of climate data, structured with a microservices-based architecture. The system performs periodic collection of meteorological data, intermediate processing, persistence, and AI-driven insight generation, providing an interactive view for the end user.}  \\[-2pt]
\multicolumn{2}{@{}X@{}}{\vspace{1pt}\textbf{Stack}: Python, Go, TypeScript, NestJS, React, Vite, MongoDB, RabbitMQ, Ollama, Docker.}\\[1pt]
\end{tabularx}

\begin{tabularx}{\linewidth}{ @{}l r@{} }
\textbf{Project Kiwi} & \hfill \href{https://helpmeagain.itch.io/project-kiwi}{Demo} — \href{https://github.com/helpmeagain/project-kiwi}{Repository} \\[3.75pt]
\multicolumn{2}{@{}X@{}}{Educational game designed to teach English with the goal of making learning more collaborative and interactive through gamification. The project was implemented with a focus on user experience, cooperative gameplay, and usability. Testing with students showed that the addition of multiplayer mode increased engagement, acceptance, and interest in the content.}  \\[-2pt]
\multicolumn{2}{@{}X@{}}{\vspace{1pt}\textbf{Stack}: Godot, GDScript, Nakama, Docker.}\\[1pt]
\end{tabularx}

\begin{tabularx}{\linewidth}{ @{}l r@{} }
\textbf{Blackwell} & \hfill \href{https://github.com/helpmeagain/blackwell}{Repository} \\[3.75pt]
\multicolumn{2}{@{}X@{}}{API and mobile application developed for generating electronic physiotherapy medical records for a teaching clinic. The solution was structured using Domain-Driven Design and Clean Architecture, ensuring scalability and ease of maintenance. The project optimized patient care workflows, reduced data entry errors, and improved access to patient information.}  \\[-2pt]
\multicolumn{2}{@{}X@{}}{\vspace{1pt}\textbf{Stack}: TypeScript, NestJS, Prisma, Vitest, PostgreSQL, Redis, Docker, React Native, Expo.}\\[1pt]
\end{tabularx}

%----------------------------------------------------------------------------------------
%	PUBLICATIONS
%----------------------------------------------------------------------------------------
% \section{Publicações}
% \begin{refsection}[citations.bib]
% \nocite{*}
% \printbibliography[heading=none]
% \end{refsection}

%----------------------------------------------------------------------------------------
%	SKILLS
%----------------------------------------------------------------------------------------
\section{Skills}
\begin{tabularx}{\linewidth}{@{}l X@{}}
\textbf{Programming Languages} & \normalsize{JavaScript, TypeScript, C{\texttt{\#}}, Go, Java, Python.}\\
\textbf{Web Development} & \normalsize{NodeJS, Express, NestJS, React, NextJS, Angular, VueJS, NuxtJS, HTML, CSS, Tailwind, .NET, Entity Framework, Django, Flask, Spring Boot.}\\
\textbf{Mobile Development} & \normalsize{React Native, Expo, Ionic.}\\
\textbf{Databases}  &  \normalsize{MySQL, SQL Server, PostgreSQL, MongoDB, Redis.}\\
\textbf{Miscellaneous}  &  \normalsize{Docker, AWS, Google Cloud, Azure DevOps, Ollama, Godot.}\\  
\end{tabularx}

\vfill
\center{\footnotesize Last updated: \today}

\end{document}
